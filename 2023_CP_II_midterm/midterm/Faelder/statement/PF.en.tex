%!TEX root = problem.en.tex
\gdef\thisproblemauthor{}
\gdef\thisproblemdeveloper{}
\gdef\thisproblemorigin{}
\begin{problem}{Faelder}
{}{}
{please refer to DOM Judge}
{please refer to DOM Judge}
{}

You have an $N\times M$ chess board and a kind of special chess piece called "faelder".

The $(i, j)$ grid of the chess board is the intersection of the $i$-th row from the top and the $j$-th column from the right.

Each grid of the chess board can be occupied by at most one faelder.

However, if there's a faelder at $(i, j)$ and there's another faelder on $(i-2, j)$, $(i+2, j)$, $(i, j-2)$, or $(i,j+2)$, then they will attack each other, causing the chess board to explode.

It is also reasonable that you don't want your chess board to explode.

On the other hand, there are $c$ grids of your chess board already ocuppied by some rocks collected by your younger siblings.

What's more, you can't put faelder at those $c$ grids since it will make your siblings cry and you will be punished by your parents, just because you're a little older.

Fortunately, faelders won't attack the rocks since they're afraid of your siblings' tears, too. That is, one faelder only attacks other faelders.

To prevent more grids from occupying by your siblings, you want to put as many faelders as possible on the chess board.

What is the maximum number of faelders that can be placed on the chess board?

\InputFile
The first line of the input contains two integers $N, M$ --- the size of your chess board.

The second line of the input contains an integer $c$ --- the number of grids occupied by your siblings' rocks.

Each of the next $c$ lines contains two integers $x_i, y_i$ --- the grid $(x_i, y_i)$ is occupied.

\begin{itemize}
    \item $1 \leq N, M \leq 300$
    \item $0 \leq c \leq N\times M$
    \item $1 \leq x_i \leq N$
    \item $1 \leq y_i \leq M$
    \item The pairs $(x_i, y_i)$ are distinct.
\end{itemize}

\OutputFile
The output should contain an integer representing the maximum number of faelders that can be placed on the chess board.

\Examples

\begin{example}
\exmpfile{./probs/PF/0-01.in}{./probs/PF/0-01.out}%
\exmpfile{./probs/PF/0-02.in}{./probs/PF/0-02.out}%
\end{example}

\end{problem}
